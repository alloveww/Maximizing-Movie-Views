% Options for packages loaded elsewhere
\PassOptionsToPackage{unicode}{hyperref}
\PassOptionsToPackage{hyphens}{url}
%
\documentclass[
]{article}
\usepackage{lmodern}
\usepackage{amsmath}
\usepackage{ifxetex,ifluatex}
\ifnum 0\ifxetex 1\fi\ifluatex 1\fi=0 % if pdftex
  \usepackage[T1]{fontenc}
  \usepackage[utf8]{inputenc}
  \usepackage{textcomp} % provide euro and other symbols
  \usepackage{amssymb}
\else % if luatex or xetex
  \usepackage{unicode-math}
  \defaultfontfeatures{Scale=MatchLowercase}
  \defaultfontfeatures[\rmfamily]{Ligatures=TeX,Scale=1}
\fi
% Use upquote if available, for straight quotes in verbatim environments
\IfFileExists{upquote.sty}{\usepackage{upquote}}{}
\IfFileExists{microtype.sty}{% use microtype if available
  \usepackage[]{microtype}
  \UseMicrotypeSet[protrusion]{basicmath} % disable protrusion for tt fonts
}{}
\makeatletter
\@ifundefined{KOMAClassName}{% if non-KOMA class
  \IfFileExists{parskip.sty}{%
    \usepackage{parskip}
  }{% else
    \setlength{\parindent}{0pt}
    \setlength{\parskip}{6pt plus 2pt minus 1pt}}
}{% if KOMA class
  \KOMAoptions{parskip=half}}
\makeatother
\usepackage{xcolor}
\IfFileExists{xurl.sty}{\usepackage{xurl}}{} % add URL line breaks if available
\IfFileExists{bookmark.sty}{\usepackage{bookmark}}{\usepackage{hyperref}}
\hypersetup{
  pdftitle={Math208 Final Project},
  pdfauthor={Joseph Wu, Darren Bugaresti, Anna Hayes},
  hidelinks,
  pdfcreator={LaTeX via pandoc}}
\urlstyle{same} % disable monospaced font for URLs
\usepackage[margin=1in]{geometry}
\usepackage{color}
\usepackage{fancyvrb}
\newcommand{\VerbBar}{|}
\newcommand{\VERB}{\Verb[commandchars=\\\{\}]}
\DefineVerbatimEnvironment{Highlighting}{Verbatim}{commandchars=\\\{\}}
% Add ',fontsize=\small' for more characters per line
\usepackage{framed}
\definecolor{shadecolor}{RGB}{248,248,248}
\newenvironment{Shaded}{\begin{snugshade}}{\end{snugshade}}
\newcommand{\AlertTok}[1]{\textcolor[rgb]{0.94,0.16,0.16}{#1}}
\newcommand{\AnnotationTok}[1]{\textcolor[rgb]{0.56,0.35,0.01}{\textbf{\textit{#1}}}}
\newcommand{\AttributeTok}[1]{\textcolor[rgb]{0.77,0.63,0.00}{#1}}
\newcommand{\BaseNTok}[1]{\textcolor[rgb]{0.00,0.00,0.81}{#1}}
\newcommand{\BuiltInTok}[1]{#1}
\newcommand{\CharTok}[1]{\textcolor[rgb]{0.31,0.60,0.02}{#1}}
\newcommand{\CommentTok}[1]{\textcolor[rgb]{0.56,0.35,0.01}{\textit{#1}}}
\newcommand{\CommentVarTok}[1]{\textcolor[rgb]{0.56,0.35,0.01}{\textbf{\textit{#1}}}}
\newcommand{\ConstantTok}[1]{\textcolor[rgb]{0.00,0.00,0.00}{#1}}
\newcommand{\ControlFlowTok}[1]{\textcolor[rgb]{0.13,0.29,0.53}{\textbf{#1}}}
\newcommand{\DataTypeTok}[1]{\textcolor[rgb]{0.13,0.29,0.53}{#1}}
\newcommand{\DecValTok}[1]{\textcolor[rgb]{0.00,0.00,0.81}{#1}}
\newcommand{\DocumentationTok}[1]{\textcolor[rgb]{0.56,0.35,0.01}{\textbf{\textit{#1}}}}
\newcommand{\ErrorTok}[1]{\textcolor[rgb]{0.64,0.00,0.00}{\textbf{#1}}}
\newcommand{\ExtensionTok}[1]{#1}
\newcommand{\FloatTok}[1]{\textcolor[rgb]{0.00,0.00,0.81}{#1}}
\newcommand{\FunctionTok}[1]{\textcolor[rgb]{0.00,0.00,0.00}{#1}}
\newcommand{\ImportTok}[1]{#1}
\newcommand{\InformationTok}[1]{\textcolor[rgb]{0.56,0.35,0.01}{\textbf{\textit{#1}}}}
\newcommand{\KeywordTok}[1]{\textcolor[rgb]{0.13,0.29,0.53}{\textbf{#1}}}
\newcommand{\NormalTok}[1]{#1}
\newcommand{\OperatorTok}[1]{\textcolor[rgb]{0.81,0.36,0.00}{\textbf{#1}}}
\newcommand{\OtherTok}[1]{\textcolor[rgb]{0.56,0.35,0.01}{#1}}
\newcommand{\PreprocessorTok}[1]{\textcolor[rgb]{0.56,0.35,0.01}{\textit{#1}}}
\newcommand{\RegionMarkerTok}[1]{#1}
\newcommand{\SpecialCharTok}[1]{\textcolor[rgb]{0.00,0.00,0.00}{#1}}
\newcommand{\SpecialStringTok}[1]{\textcolor[rgb]{0.31,0.60,0.02}{#1}}
\newcommand{\StringTok}[1]{\textcolor[rgb]{0.31,0.60,0.02}{#1}}
\newcommand{\VariableTok}[1]{\textcolor[rgb]{0.00,0.00,0.00}{#1}}
\newcommand{\VerbatimStringTok}[1]{\textcolor[rgb]{0.31,0.60,0.02}{#1}}
\newcommand{\WarningTok}[1]{\textcolor[rgb]{0.56,0.35,0.01}{\textbf{\textit{#1}}}}
\usepackage{graphicx}
\makeatletter
\def\maxwidth{\ifdim\Gin@nat@width>\linewidth\linewidth\else\Gin@nat@width\fi}
\def\maxheight{\ifdim\Gin@nat@height>\textheight\textheight\else\Gin@nat@height\fi}
\makeatother
% Scale images if necessary, so that they will not overflow the page
% margins by default, and it is still possible to overwrite the defaults
% using explicit options in \includegraphics[width, height, ...]{}
\setkeys{Gin}{width=\maxwidth,height=\maxheight,keepaspectratio}
% Set default figure placement to htbp
\makeatletter
\def\fps@figure{htbp}
\makeatother
\setlength{\emergencystretch}{3em} % prevent overfull lines
\providecommand{\tightlist}{%
  \setlength{\itemsep}{0pt}\setlength{\parskip}{0pt}}
\setcounter{secnumdepth}{-\maxdimen} % remove section numbering
\usepackage{booktabs}
\usepackage{longtable}
\usepackage{array}
\usepackage{multirow}
\usepackage{wrapfig}
\usepackage{float}
\usepackage{colortbl}
\usepackage{pdflscape}
\usepackage{tabu}
\usepackage{threeparttable}
\usepackage{threeparttablex}
\usepackage[normalem]{ulem}
\usepackage{makecell}
\usepackage{xcolor}
\ifluatex
  \usepackage{selnolig}  % disable illegal ligatures
\fi

\title{Math208 Final Project}
\author{Joseph Wu, Darren Bugaresti, Anna Hayes}
\date{12/1/2021}

\begin{document}
\maketitle

\begin{Shaded}
\begin{Highlighting}[]
\CommentTok{\#drop na and select only movies}
\NormalTok{project}\OtherTok{\textless{}{-}}\FunctionTok{read.csv}\NormalTok{(}\StringTok{"Final\_Project\_FlixGem.csv"}\NormalTok{)}\SpecialCharTok{\%\textgreater{}\%}\NormalTok{drop\_na}\SpecialCharTok{\%\textgreater{}\%}\NormalTok{as\_tibble}\SpecialCharTok{\%\textgreater{}\%}\FunctionTok{filter}\NormalTok{(Series.or.Movie}\SpecialCharTok{==}\StringTok{"Movie"}\NormalTok{)}
\end{Highlighting}
\end{Shaded}

\hypertarget{task-1-data-wrangling-and-exploratory-data-analyses}{%
\subsection{\texorpdfstring{\textbf{Task 1} : Data Wrangling and
Exploratory Data
Analyses}{Task 1 : Data Wrangling and Exploratory Data Analyses}}\label{task-1-data-wrangling-and-exploratory-data-analyses}}

\hypertarget{a.-does-the-hidden-gem-score-seems-to-be-associated-to-the-runtime-category-or-the-languages-used-in-the-film-explain-briefly-the-reasons-behind-your-assessment.-hint-you-may-need-to-do-some-re-coding-of-one-or-both-of-these-variables.-any-reasonable-re-coding-is-fine-just-be-sure-to-be-clear-what-youve-done.}{%
\subsubsection{a. Does the Hidden Gem Score seems to be associated to
the Runtime Category or the languages used in the film? Explain briefly
the reasons behind your assessment. Hint: You may need to do some
re-coding of one or both of these variables. Any reasonable re-coding is
fine, just be sure to be clear what you've
done.}\label{a.-does-the-hidden-gem-score-seems-to-be-associated-to-the-runtime-category-or-the-languages-used-in-the-film-explain-briefly-the-reasons-behind-your-assessment.-hint-you-may-need-to-do-some-re-coding-of-one-or-both-of-these-variables.-any-reasonable-re-coding-is-fine-just-be-sure-to-be-clear-what-youve-done.}}

\begin{Shaded}
\begin{Highlighting}[]
\CommentTok{\#group Hidden.Gem.score into 17 subgroups }
\NormalTok{lbls }\OtherTok{\textless{}{-}}
  \FunctionTok{c}\NormalTok{( }\StringTok{\textquotesingle{}0.5{-}1.0\textquotesingle{}}\NormalTok{, }\StringTok{\textquotesingle{}1.0{-}1.5\textquotesingle{}}\NormalTok{,  }\StringTok{\textquotesingle{}1.5{-}2.0\textquotesingle{}}\NormalTok{,  }\StringTok{\textquotesingle{}2.0{-}2.5\textquotesingle{}}\NormalTok{, }\StringTok{\textquotesingle{}2.5{-}3.0\textquotesingle{}}\NormalTok{,}
     \StringTok{\textquotesingle{}3.0{-}3.5\textquotesingle{}}\NormalTok{, }\StringTok{\textquotesingle{}3.5{-}4.0\textquotesingle{}}\NormalTok{,  }\StringTok{\textquotesingle{}4.0{-}4.5\textquotesingle{}}\NormalTok{,  }\StringTok{\textquotesingle{}4.5{-}5.0\textquotesingle{}}\NormalTok{, }\StringTok{\textquotesingle{}5.0{-}5.5\textquotesingle{}}\NormalTok{,}
     \StringTok{\textquotesingle{}5.5{-}6.0\textquotesingle{}}\NormalTok{, }\StringTok{\textquotesingle{}6.0{-}6.5\textquotesingle{}}\NormalTok{,  }\StringTok{\textquotesingle{}6.5{-}7.0\textquotesingle{}}\NormalTok{,  }\StringTok{\textquotesingle{}7.0{-}7.5\textquotesingle{}}\NormalTok{, }\StringTok{\textquotesingle{}7.5{-}8.0\textquotesingle{}}\NormalTok{,}
     \StringTok{\textquotesingle{}8.0{-}8.5\textquotesingle{}}\NormalTok{, }\StringTok{\textquotesingle{}8.5{-}9.0\textquotesingle{}}\NormalTok{)}
\NormalTok{breaks }\OtherTok{=} \FunctionTok{c}\NormalTok{(}\FloatTok{0.5}\NormalTok{ ,}\FloatTok{1.0}\NormalTok{, }\FloatTok{1.5}\NormalTok{ ,}\FloatTok{2.0}\NormalTok{ ,}\FloatTok{2.5}\NormalTok{ ,}\FloatTok{3.0}\NormalTok{ ,}\FloatTok{3.5}\NormalTok{ ,}\FloatTok{4.0}\NormalTok{ ,}
           \FloatTok{4.5}\NormalTok{ ,}\FloatTok{5.0}\NormalTok{ ,}\FloatTok{5.5}\NormalTok{ ,}\FloatTok{6.0}\NormalTok{ ,}\FloatTok{6.5}\NormalTok{ ,}\FloatTok{7.0}\NormalTok{ ,}\FloatTok{7.5}\NormalTok{ ,}\FloatTok{8.0}\NormalTok{ ,}
           \FloatTok{8.5}\NormalTok{ ,}\FloatTok{9.0}\NormalTok{)}
\NormalTok{group\_HGS2 }\OtherTok{\textless{}{-}}\NormalTok{project }\SpecialCharTok{\%\textgreater{}\%} \FunctionTok{mutate}\NormalTok{(}\AttributeTok{HGS\_group =} \FunctionTok{cut}\NormalTok{(}
\NormalTok{             Hidden.Gem.Score,}
             \AttributeTok{breaks =}\NormalTok{ breaks,}
             \AttributeTok{right =}\NormalTok{ T,}
             \AttributeTok{labels =}\NormalTok{ lbls ))}

\NormalTok{group\_HGS2}\OtherTok{\textless{}{-}}\NormalTok{group\_HGS2}\SpecialCharTok{\%\textgreater{}\%}\FunctionTok{ungroup}\NormalTok{()}\SpecialCharTok{\%\textgreater{}\%}\FunctionTok{group\_by}\NormalTok{(HGS\_group)}\SpecialCharTok{\%\textgreater{}\%}\FunctionTok{mutate}\NormalTok{(}\AttributeTok{Count=}\FunctionTok{n}\NormalTok{())}
\NormalTok{group\_HGS2}\OtherTok{\textless{}{-}}\NormalTok{group\_HGS2}\SpecialCharTok{\%\textgreater{}\%}\FunctionTok{group\_by}\NormalTok{(HGS\_group) }\SpecialCharTok{\%\textgreater{}\%} \FunctionTok{mutate}\NormalTok{(}\AttributeTok{Proportion =}\NormalTok{Count }\SpecialCharTok{/} \FunctionTok{sum}\NormalTok{(Count))}

  
\FunctionTok{ggplot}\NormalTok{(group\_HGS2, }\FunctionTok{aes}\NormalTok{(}\AttributeTok{x =}\NormalTok{ HGS\_group, }\AttributeTok{y=}\NormalTok{ Proportion, }\AttributeTok{fill =}\NormalTok{ Runtime, }\AttributeTok{width =}\NormalTok{ .}\DecValTok{8}\NormalTok{)) }\SpecialCharTok{+} 
  \FunctionTok{geom\_bar}\NormalTok{(}\AttributeTok{stat =} \StringTok{"identity"}\NormalTok{) }\SpecialCharTok{+}
  \FunctionTok{scale\_fill\_viridis\_d}\NormalTok{() }\SpecialCharTok{+} \FunctionTok{xlab}\NormalTok{(}\StringTok{"Hidden.Gem.Score"}\NormalTok{)  }\SpecialCharTok{+}
  \FunctionTok{theme}\NormalTok{(}\AttributeTok{axis.text.x =} \FunctionTok{element\_text}\NormalTok{(}\AttributeTok{angle =} \DecValTok{90}\NormalTok{)) }\SpecialCharTok{+} \FunctionTok{labs}\NormalTok{(}\AttributeTok{title =} \StringTok{"Hidden Gem Score vs Runtime"}\NormalTok{)}
\end{Highlighting}
\end{Shaded}

\includegraphics{project-3_files/figure-latex/a-1.pdf}

\begin{Shaded}
\begin{Highlighting}[]
\CommentTok{\#too many languages, manipulation with the languages here..}
\NormalTok{temp }\OtherTok{\textless{}{-}} \FunctionTok{drop\_na}\NormalTok{(project)[}\FunctionTok{c}\NormalTok{(}\StringTok{"Languages"}\NormalTok{)] }\SpecialCharTok{\%\textgreater{}\%} \FunctionTok{unique}\NormalTok{()}
\NormalTok{input\_fun }\OtherTok{\textless{}{-}}
  \ControlFlowTok{function}\NormalTok{(x) \{}\CommentTok{\#helper function 1: split the string}
    \FunctionTok{str\_split}\NormalTok{(x, }\StringTok{", "}\NormalTok{)}
\NormalTok{  \}}
\NormalTok{unique\_language }\OtherTok{\textless{}{-}}
  \FunctionTok{lapply}\NormalTok{(temp, input\_fun) }\SpecialCharTok{\%\textgreater{}\%} \FunctionTok{unlist}\NormalTok{() }\SpecialCharTok{\%\textgreater{}\%} \FunctionTok{unique}\NormalTok{()   }
\FunctionTok{length}\NormalTok{(unique\_language)}
\end{Highlighting}
\end{Shaded}

\begin{verbatim}
## [1] 121
\end{verbatim}

\begin{Shaded}
\begin{Highlighting}[]
\CommentTok{\#there are total 121 unique languages, now we group the languages}
\CommentTok{\#select top six most spoken languages into 6 different groups and rest languages into a single group}
\NormalTok{Lgroup1 }\OtherTok{\textless{}{-}} \FunctionTok{c}\NormalTok{(}\StringTok{"English"}\NormalTok{)}
\NormalTok{Lgroup2 }\OtherTok{\textless{}{-}} \FunctionTok{c}\NormalTok{(}\StringTok{"Spanish"}\NormalTok{)}
\NormalTok{Lgroup3 }\OtherTok{\textless{}{-}} \FunctionTok{c}\NormalTok{(}\StringTok{"French"}\NormalTok{)}
\NormalTok{Lgroup4 }\OtherTok{\textless{}{-}} \FunctionTok{c}\NormalTok{(}\StringTok{"Hindi"}\NormalTok{)}
\NormalTok{Lgroup5 }\OtherTok{\textless{}{-}} \FunctionTok{c}\NormalTok{(}\StringTok{"Chinese"}\NormalTok{)}
\NormalTok{Lgroup6 }\OtherTok{\textless{}{-}} \FunctionTok{c}\NormalTok{(}\StringTok{"Arabic"}\NormalTok{)}
\NormalTok{Lgroup7 }\OtherTok{\textless{}{-}}
\NormalTok{  unique\_language[}\SpecialCharTok{!}\NormalTok{unique\_language }\SpecialCharTok{\%in\%} \FunctionTok{c}\NormalTok{(}\StringTok{"English"}\NormalTok{, }\StringTok{"Spanish"}\NormalTok{, }\StringTok{"French"}\NormalTok{, }\StringTok{"Hindi"}\NormalTok{, }\StringTok{"Chinese"}\NormalTok{, }\StringTok{"Arabic"}\NormalTok{)]}
\NormalTok{input\_fun2 }\OtherTok{\textless{}{-}}
  \ControlFlowTok{function}\NormalTok{(x, g1, g2, g3, g4, g5, g6, g7) \{}
    \CommentTok{\# helper function 2: x is the original language group and return the new language group!}
\NormalTok{    TF\_array7 }\OtherTok{\textless{}{-}} \FunctionTok{input\_fun}\NormalTok{(x) }\SpecialCharTok{\%\textgreater{}\%} \FunctionTok{unlist}\NormalTok{() }\SpecialCharTok{\%in\%}\NormalTok{ g7}
\NormalTok{    TF\_array6 }\OtherTok{\textless{}{-}} \FunctionTok{input\_fun}\NormalTok{(x) }\SpecialCharTok{\%\textgreater{}\%} \FunctionTok{unlist}\NormalTok{() }\SpecialCharTok{\%in\%}\NormalTok{ g6}
\NormalTok{    TF\_array5 }\OtherTok{\textless{}{-}} \FunctionTok{input\_fun}\NormalTok{(x) }\SpecialCharTok{\%\textgreater{}\%} \FunctionTok{unlist}\NormalTok{() }\SpecialCharTok{\%in\%}\NormalTok{ g5}
\NormalTok{    TF\_array4 }\OtherTok{\textless{}{-}} \FunctionTok{input\_fun}\NormalTok{(x) }\SpecialCharTok{\%\textgreater{}\%} \FunctionTok{unlist}\NormalTok{() }\SpecialCharTok{\%in\%}\NormalTok{ g4}
\NormalTok{    TF\_array3 }\OtherTok{\textless{}{-}} \FunctionTok{input\_fun}\NormalTok{(x) }\SpecialCharTok{\%\textgreater{}\%} \FunctionTok{unlist}\NormalTok{() }\SpecialCharTok{\%in\%}\NormalTok{ g3}
\NormalTok{    TF\_array2 }\OtherTok{\textless{}{-}} \FunctionTok{input\_fun}\NormalTok{(x) }\SpecialCharTok{\%\textgreater{}\%} \FunctionTok{unlist}\NormalTok{() }\SpecialCharTok{\%in\%}\NormalTok{ g2}
\NormalTok{    TF\_array1 }\OtherTok{\textless{}{-}} \FunctionTok{input\_fun}\NormalTok{(x) }\SpecialCharTok{\%\textgreater{}\%} \FunctionTok{unlist}\NormalTok{() }\SpecialCharTok{\%in\%}\NormalTok{ g1}
    \ControlFlowTok{if}\NormalTok{ (}\ConstantTok{TRUE} \SpecialCharTok{\%in\%}\NormalTok{ TF\_array7) \{ }\FunctionTok{return}\NormalTok{(}\StringTok{"Rare\_Language"}\NormalTok{) \}}
    \ControlFlowTok{if}\NormalTok{ (}\ConstantTok{TRUE} \SpecialCharTok{\%in\%}\NormalTok{ TF\_array6) \{ }\FunctionTok{return}\NormalTok{(}\StringTok{"Arabic"}\NormalTok{)\}}
    \ControlFlowTok{if}\NormalTok{ (}\ConstantTok{TRUE} \SpecialCharTok{\%in\%}\NormalTok{ TF\_array5) \{ }\FunctionTok{return}\NormalTok{(}\StringTok{"Chinese"}\NormalTok{)\}}
    \ControlFlowTok{if}\NormalTok{ (}\ConstantTok{TRUE} \SpecialCharTok{\%in\%}\NormalTok{ TF\_array4) \{ }\FunctionTok{return}\NormalTok{(}\StringTok{"Hindi"}\NormalTok{)\}}
    \ControlFlowTok{if}\NormalTok{ (}\ConstantTok{TRUE} \SpecialCharTok{\%in\%}\NormalTok{ TF\_array3) \{ }\FunctionTok{return}\NormalTok{(}\StringTok{"French"}\NormalTok{)\}}
    \ControlFlowTok{if}\NormalTok{ (}\ConstantTok{TRUE} \SpecialCharTok{\%in\%}\NormalTok{ TF\_array2) \{ }\FunctionTok{return}\NormalTok{(}\StringTok{"Spanish"}\NormalTok{) \}}
    \ControlFlowTok{if}\NormalTok{ (}\ConstantTok{TRUE} \SpecialCharTok{\%in\%}\NormalTok{ TF\_array1) \{ }\FunctionTok{return}\NormalTok{(}\StringTok{"English"}\NormalTok{) \}}
\NormalTok{  \}}
\ControlFlowTok{for}\NormalTok{ (i }\ControlFlowTok{in} \DecValTok{1}\SpecialCharTok{:}\FunctionTok{length}\NormalTok{(group\_HGS2}\SpecialCharTok{$}\NormalTok{Languages)) \{}
  \CommentTok{\#loop around the char vector and change every single cell!}
\NormalTok{ group\_HGS2}\SpecialCharTok{$}\NormalTok{Languages[i] }\OtherTok{\textless{}{-}}
    \FunctionTok{input\_fun2}\NormalTok{(}
\NormalTok{    group\_HGS2}\SpecialCharTok{$}\NormalTok{Languages[i], Lgroup1, Lgroup2,Lgroup3, Lgroup4, Lgroup5,}
\NormalTok{      Lgroup6, Lgroup7}
\NormalTok{    )}
\NormalTok{\}}
\CommentTok{\#languages manipulation ends here, now plot!}
\FunctionTok{ggplot}\NormalTok{(group\_HGS2, }\FunctionTok{aes}\NormalTok{(}\AttributeTok{x =}\NormalTok{ HGS\_group, }\AttributeTok{y=}\NormalTok{ Proportion, }\AttributeTok{fill =}\NormalTok{ Languages, }\AttributeTok{width =}\NormalTok{ .}\DecValTok{8}\NormalTok{)) }\SpecialCharTok{+} \FunctionTok{geom\_bar}\NormalTok{(}\AttributeTok{stat =} \StringTok{"identity"}\NormalTok{) }\SpecialCharTok{+}
  \FunctionTok{scale\_fill\_viridis\_d}\NormalTok{() }\SpecialCharTok{+}  \FunctionTok{xlab}\NormalTok{(}\StringTok{"Hidden.Gem.Score"}\NormalTok{) }\SpecialCharTok{+} 
  \FunctionTok{theme}\NormalTok{(}\AttributeTok{axis.text.x =} \FunctionTok{element\_text}\NormalTok{(}\AttributeTok{angle =} \DecValTok{90}\NormalTok{)) }\SpecialCharTok{+} \FunctionTok{labs}\NormalTok{(}\AttributeTok{title =} \StringTok{"Hidden Gem Score vs Langauges"}\NormalTok{)}
\end{Highlighting}
\end{Shaded}

\includegraphics{project-3_files/figure-latex/a-2.pdf}

The Hidden Gem Score seems to be most associated to the Runtime, since
the proportions fluctuate more with the levels of the runtime category.
With languages, the proportions of languages for each Hidden Gem Score
interval are about the same. Meanwhile, with Runtime, we can see how the
proportions of runtime vary with the hidden gem score, with films over 2
hours increasing in Hidden Gem Score

\newpage

\hypertarget{b.-do-any-of-the-three-review-site-scores-imdb-rotten-tomatoes-metacritic-seem-to-be-strongly-or-weakly-correlated-with-the-hidden-gem-scores-explain-briefly-the-reasons-behind-your-assessment-and-the-nature-of-those-associations.}{%
\subsubsection{b. Do any of the three review site scores (IMDb, Rotten
Tomatoes, Metacritic) seem to be strongly or weakly correlated with the
Hidden Gem Scores? Explain briefly the reasons behind your assessment
and the nature of those
associations.}\label{b.-do-any-of-the-three-review-site-scores-imdb-rotten-tomatoes-metacritic-seem-to-be-strongly-or-weakly-correlated-with-the-hidden-gem-scores-explain-briefly-the-reasons-behind-your-assessment-and-the-nature-of-those-associations.}}

\begin{Shaded}
\begin{Highlighting}[]
\FunctionTok{par}\NormalTok{(}\AttributeTok{mfrow=}\FunctionTok{c}\NormalTok{(}\DecValTok{3}\NormalTok{,}\DecValTok{1}\NormalTok{))}
\FunctionTok{ggplot}\NormalTok{(project,}\FunctionTok{aes}\NormalTok{(}\AttributeTok{x=}\NormalTok{Hidden.Gem.Score,}\AttributeTok{y=}\NormalTok{Rotten.Tomatoes.Score))}\SpecialCharTok{+}\FunctionTok{geom\_bin2d}\NormalTok{(}\AttributeTok{bins=}\DecValTok{100}\NormalTok{)}\SpecialCharTok{+}\FunctionTok{geom\_smooth}\NormalTok{() }\SpecialCharTok{+} \FunctionTok{labs}\NormalTok{(}\AttributeTok{title =} \StringTok{"Hidden Gem Score vs Rotten Tomatoes Score"}\NormalTok{, }\AttributeTok{x =} \StringTok{"Hidden Gem Score"}\NormalTok{, }\AttributeTok{y =} \StringTok{"Rotten Tomatoes Score"}\NormalTok{)}
\end{Highlighting}
\end{Shaded}

\includegraphics{project-3_files/figure-latex/b-1.pdf}

\begin{Shaded}
\begin{Highlighting}[]
\FunctionTok{ggplot}\NormalTok{(project,}\FunctionTok{aes}\NormalTok{(}\AttributeTok{x=}\NormalTok{Hidden.Gem.Score,}\AttributeTok{y=}\NormalTok{Metacritic.Score))}\SpecialCharTok{+}\FunctionTok{geom\_bin2d}\NormalTok{(}\AttributeTok{bins=}\DecValTok{100}\NormalTok{)}\SpecialCharTok{+}\FunctionTok{geom\_smooth}\NormalTok{()}\SpecialCharTok{+} \FunctionTok{labs}\NormalTok{(}\AttributeTok{title =} \StringTok{"Hidden Gem Score vs Metacritic Score"}\NormalTok{, }\AttributeTok{x =} \StringTok{"Hidden Gem Score"}\NormalTok{, }\AttributeTok{y =} \StringTok{"Metacritic"}\NormalTok{)}
\end{Highlighting}
\end{Shaded}

\includegraphics{project-3_files/figure-latex/b-2.pdf}

\begin{Shaded}
\begin{Highlighting}[]
\FunctionTok{ggplot}\NormalTok{(project,}\FunctionTok{aes}\NormalTok{(}\AttributeTok{x=}\NormalTok{Hidden.Gem.Score,}\AttributeTok{y=}\NormalTok{IMDb.Score))}\SpecialCharTok{+}\FunctionTok{geom\_bin2d}\NormalTok{(}\AttributeTok{bins=}\DecValTok{100}\NormalTok{)}\SpecialCharTok{+}\FunctionTok{geom\_smooth}\NormalTok{()}\SpecialCharTok{+} \FunctionTok{labs}\NormalTok{(}\AttributeTok{title =} \StringTok{"Hidden Gem Score vs IMDb Score"}\NormalTok{, }\AttributeTok{x =} \StringTok{"Hidden Gem Score"}\NormalTok{, }\AttributeTok{y =} \StringTok{"IMDb Score"}\NormalTok{)}
\end{Highlighting}
\end{Shaded}

\includegraphics{project-3_files/figure-latex/b-3.pdf}

Based on the smoothing lines, we can see trends in the data relating the
Hidden Gem Score to other scores and can conclude that the Rotten
Tomatoes Score is strongly related to Hidden Gem Score as most of the
data points are closely compact to the line, which has a steep slope.
The line showing the relationship between the IMDB Score and the Hidden
Gem Score curves more, the slope is not as steep towards the middle, and
the data points are further spread out, meaning the two are more weakly
correlated from the above three graphs. However, for all three graphs,
the scores are not strongly associated with high Hidden Gem Scores
(scores \textgreater5.0).

\newpage

\hypertarget{c.-the-company-has-a-theory-that-people-are-becoming-more-acceptable-of-longer-movies-because-they-can-watch-them-at-home-on-netflix-and-other-content-collecting-sites.-do-you-notice-any-trend-over-time-in-the-hidden-gem-scores-by-category-of-runtime-length-explain-briefly-the-reasons-behind-your-assessment.}{%
\subsubsection{c.~The company has a theory that people are becoming more
acceptable of longer movies because they can watch them at home on
Netflix and other content-collecting sites. Do you notice any trend over
time in the Hidden Gem Scores by category of RunTime Length? Explain
briefly the reasons behind your
assessment.}\label{c.-the-company-has-a-theory-that-people-are-becoming-more-acceptable-of-longer-movies-because-they-can-watch-them-at-home-on-netflix-and-other-content-collecting-sites.-do-you-notice-any-trend-over-time-in-the-hidden-gem-scores-by-category-of-runtime-length-explain-briefly-the-reasons-behind-your-assessment.}}

\begin{Shaded}
\begin{Highlighting}[]
\CommentTok{\# x{-}axis:time, }
\CommentTok{\# y{-}axis: HGS, }
\CommentTok{\# three graphs corresponding to the different Runtime}

\CommentTok{\# by\_date\_tbl\textless{}{-}group\_HGS2\%\textgreater{}\%mutate(date\_only=date(Release.Date))\%\textgreater{}\%group\_by(date\_only)\%\textgreater{}\%summarise(count=n())}

\FunctionTok{ggplot}\NormalTok{(project, }\FunctionTok{aes}\NormalTok{(}\AttributeTok{x =}\NormalTok{ Release.Date, }\AttributeTok{y =}\NormalTok{ Hidden.Gem.Score)) }\SpecialCharTok{+} 
  \FunctionTok{geom\_point}\NormalTok{(}\FunctionTok{aes}\NormalTok{(}\AttributeTok{color =}\NormalTok{ Runtime)) }\SpecialCharTok{+} \FunctionTok{facet\_grid}\NormalTok{(}\StringTok{\textquotesingle{}Runtime\textquotesingle{}}\NormalTok{) }\SpecialCharTok{+} \FunctionTok{labs}\NormalTok{(}\AttributeTok{title =} \StringTok{"Hidden Gem Score Over Time by Runtime"}\NormalTok{, }\AttributeTok{x =} \StringTok{"Release Date"}\NormalTok{, }\AttributeTok{y =} \StringTok{"Hidden Gem Score"}\NormalTok{)}
\end{Highlighting}
\end{Shaded}

\includegraphics{project-3_files/figure-latex/c-1.pdf}

Within the last decade, longer movies (with runtimes \textgreater{} two
hours) are being given higher Hidden Gem Scores. On the scatterplot, it
may be observed that before the year 2000, the highest hidden gem score
that movies over 2 hours aschieved was about a 5.0, whereas over a dozen
movies have been plotted at over the 5.0 mark since 2000. However, this
could be associated with the fact that there are more longer movies
being made over time, as the number of points plotted increases in that
same time frame. Furthermore, 1-2 hour long movies have also increased
in number since 2000, as seen on the graph and also have consistently
been placed at over a 5.0 hidden gem score after not receiving that high
of a score before 2000.

\newpage

\hypertarget{task2-regression-tree}{%
\subsection{\texorpdfstring{\textbf{Task2} : Regression
Tree}{Task2 : Regression Tree}}\label{task2-regression-tree}}

\begin{Shaded}
\begin{Highlighting}[]
\CommentTok{\#select predictors }

\NormalTok{task2\_df}\OtherTok{\textless{}{-}}\NormalTok{group\_HGS2}\SpecialCharTok{\%\textgreater{}\%}\FunctionTok{ungroup}\NormalTok{()}\SpecialCharTok{\%\textgreater{}\%}\FunctionTok{select}\NormalTok{(}\StringTok{"Languages"}\NormalTok{,}\StringTok{"Runtime"}\NormalTok{,}\StringTok{"Metacritic.Score"}\NormalTok{,}\StringTok{"IMDb.Score"}\NormalTok{,}\StringTok{"Rotten.Tomatoes.Score"}\NormalTok{,}\StringTok{"Hidden.Gem.Score"}\NormalTok{)}

\NormalTok{reg\_tree}\OtherTok{\textless{}{-}}\FunctionTok{rpart}\NormalTok{(Hidden.Gem.Score }\SpecialCharTok{\textasciitilde{}}\NormalTok{.,}
                \AttributeTok{data=}\NormalTok{task2\_df,}
                \AttributeTok{method=}\StringTok{"anova"} 
\NormalTok{)}


\FunctionTok{fancyRpartPlot}\NormalTok{(reg\_tree)}
\end{Highlighting}
\end{Shaded}

\includegraphics{project-3_files/figure-latex/task 2-1.pdf}

\begin{Shaded}
\begin{Highlighting}[]
\CommentTok{\#rpart.plot(reg\_tree1)}
\end{Highlighting}
\end{Shaded}

Based on the regression tree, the most important feature for predicting
the Hidden Gem Score appears to be the Rotten Tomatoes Score. This is
because Rotten.Tomatoes have the same class as response vector
(Hidden.Gem.Score) and rpart function make an intelligent guess for
splitting the data.

\newpage

\end{document}
